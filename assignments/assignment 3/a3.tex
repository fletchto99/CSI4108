\documentclass[fleqn, 12pt]{article}

% packages %
\usepackage[includeheadfoot,headheight=15pt,margin=0.5in,left=1in,right=1in,headsep=10pt]{geometry} % page margins %
\usepackage{mathtools, amssymb} % math %
\usepackage{tabularx, multirow} % tables %
\usepackage[outputdir=.latexcache]{minted} % code %
\usepackage{graphicx} % graphics %
\usepackage{enumerate} % lists %
\usepackage{adjustbox} % images %
\usepackage[T1]{fontenc} % fonts %
\usepackage[protrusion=true,expansion=true]{microtype} % font clarity %
\usepackage{fancyhdr} % headers and footers %
\usepackage{lastpage} % reference page numbers %
\usepackage{color} % colors for code %
\usepackage{tikz} % for graphs %
\usepackage{soul} % for strikethroughout %
\usepackage{upquote} % Fix single quotes %
\usepackage{etoolbox} % Conditional checks %
\usepackage{hyperref} % Hyperlinks %
\usepackage{indentfirst} % fix indentation - only for essays %
\usepackage[figure,table]{totalcount} % For counting tables and figures %
\usepackage[utf8]{inputenc} % Encode as UTF-8 %
\usepackage{biblatex} % References %
\addbibresource{references.bib} % bib source %

% Document details %
\newcommand{\university}{University of Ottawa}
\newcommand{\name}{Matthew Langlois}
\newcommand{\studentNumber}{7731813}
\newcommand{\semester}{Fall 2017}
\newcommand{\assignmentType}{Assignment}
\newcommand{\assignmentNumber}{3}
\newcommand{\dueDate}{Nov 14}
\newcommand{\courseCode}{CSI4108}
\newcommand{\courseTitle}{Intro to Cryptography}
\newcommand{\essayTitle}{<Title>} % only for essays %
\newcommand{\essaySubtitle}{<subtitle>} % only for essays %
\newcommand{\essayAbstract}{} % Only for essays -- leave empty for no abstract %

% Center image and diagrams %
\adjustboxset*{center}

% Code settings %
\setminted{
    fontfamily=tt,
    gobble=0,
    frame=single,
    funcnamehighlighting=true,
    tabsize=4,
    obeytabs=false,
    mathescape=false
    samepage=false,
    showspaces=false,
    showtabs =false,
    texcl=false,
    breaklines=true,
}

% inline code %
\definecolor{codegray}{gray}{0.9}
\newcommand{\code}[1]{\colorbox{codegray}{\texttt{#1}}}

% Code from tile %
\newcommand{\codefile}{\inputminted}

% Graphing stuff %
\usetikzlibrary{arrows.meta}
\usetikzlibrary{positioning}
\usetikzlibrary{matrix}
\usetikzlibrary{automata}

% Define ceiling and floor functions %
\DeclarePairedDelimiter\ceil{\lceil}{\rceil}
\DeclarePairedDelimiter\floor{\lfloor}{\rfloor}

% Create set compliment command %
\newcommand{\setcomp}[1]{{#1}^{\mathsf{c}}}

% Create logic command aliases %
\newcommand{\limplies}{\rightarrow}
\newcommand{\nequiv}{\not\equiv}
\newcommand{\liff}{\leftrightarrow}

% first page header & footer %
\fancypagestyle{assignment}{
    \fancyhf{}
    \renewcommand{\footrulewidth}{0.1mm}
    \fancyfoot[R]{\assignmentType\text{ }\assignmentNumber}
    \fancyfoot[C]{\thepage \hspace{1pt} of \pageref*{LastPage}}
    \fancyfoot[L]{\courseCode\text{ }\semester}
    \renewcommand{\headrulewidth}{0mm}
}

% Frontmatter for essay page numbering%
\fancypagestyle{frontmatter}{
    \fancyhf{}
    \renewcommand{\footrulewidth}{0.1mm}
    \fancyfoot[R]{\assignmentType\text{ }\assignmentNumber}
    \fancyfoot[C]{\thepage \hspace{1pt} of \pageref*{EndFrontMatter}}
    \fancyfoot[L]{\courseCode\text{ }\semester}
    \fancyhead[L]{\name}
    \fancyhead[R]{\studentNumber}
}

% Essay body page numbering %
\fancypagestyle{body}{
    \fancyhf{}
    \renewcommand{\footrulewidth}{0.1mm}
    \fancyfoot[R]{\assignmentType\text{ }\assignmentNumber}
    \fancyfoot[C]{\thepage \hspace{1pt} of \pageref*{LastPage}}
    \fancyfoot[L]{\courseCode\text{ }\semester}
    \fancyhead[L]{\name}
    \fancyhead[R]{\studentNumber}
}

% Page header and footers %
\fancyhf{}
\renewcommand{\footrulewidth}{0.1mm}
\fancyfoot[R]{\assignmentType\text{ }\assignmentNumber}
\fancyfoot[C]{\thepage \hspace{1pt} of \pageref*{LastPage}}
\fancyfoot[L]{\courseCode\text{ }\semester}
\fancyhead[L]{\name}
\fancyhead[R]{\studentNumber}

% Apply headers & footer page style %
\pagestyle{fancy}

% Assignment first page header %
\newcommand{\beginassignemnt}{
    % Prevent paragraph indents, this isn't an English assignment! %
    \newlength\tindent
    \setlength{\tindent}{\parindent}
    \setlength{\parindent}{0pt}

    \thispagestyle{assignment}
    \noindent
    \courseTitle \hfill \university\\
    \courseCode \hfill \semester
    \begin{center}
        \textbf{\assignmentType\text{ }\#\assignmentNumber}\\
        \name \hspace{1pt} - \studentNumber\\
        \dueDate\\
    \end{center}
    \vspace{6pt}
    \hrule
    \vspace{1.5\headsep}
}

% Essay titlepage stuff %
\newcommand{\beginessay}{
    % Load all citations %
    \nocite{*}

    % Special numbering for front matter %
    \pagestyle{frontmatter}
    \pagenumbering{roman}

    % Titlepage stuff %
    \begin{center}
        \normalsize
        \textsc{\university}\\[5cm]
        \LARGE \textbf{\MakeUppercase{\essayTitle}}\\[0.5cm]
        \large \text{ }\essaySubtitle\text{ }\\[10cm] % blank \texts are used for empty subtitles %
        \normalsize
        \textsc{\name}\\
        \textsc{\studentNumber}\\
        \textsc{\courseCode}\\
        \textsc{\semester}\\
        \textsc{\dueDate}
    \end{center}
    \thispagestyle{empty}
    % End title page stuff %

    % Table of Contents %
    \newpage
    \tableofcontents
    \newpage

    % If more than 1 table/figure show appropriate lists %
    \iftotalfigures
        \addcontentsline{toc}{section}{\listfigurename}
        \listoffigures
    \fi
    \iftotaltables
        \addcontentsline{toc}{section}{\listtablename}
        \listoftables
    \fi

    % Display an abstract if the abstract isn't empty %
    \ifdefempty{\essayAbstract}{}{
        \newpage
        \addcontentsline{toc}{section}{Abstract}
        \begin{abstract}
            \essayAbstract
        \end{abstract}

    }
    \label{EndFrontMatter}
    \newpage

    % Reset page numbering %
    \pagenumbering{arabic}
    \pagestyle{body}
}

% Update the bibliography command to add its self to the references
\let\oldprintbib\printbibliography
\renewcommand{\printbibliography}{
    \newpage
    \oldprintbib
    \addcontentsline{toc}{section}{References}
    \newpage
}

% Fixes a Pygments bug in ES6 -- Thanks ShareLatex! %
% \makeatletter
% \expandafter\def\csname PYGdefault@tok@err\endcsname{\def\PYGdefault@bc##1{{\strut ##1}}}
% \makeatother

% Begin the document %
\begin{document}

% makes the header for assignment/titlepage for essay %
% \beginessay
\beginassignemnt


\section*{Question 1}

Using the following properties it becomes trivial to compute $\phi(n)$:

\begin{enumerate}[1)]
    \item $\phi(mn) = \phi(m)\phi(n)$
    \item $\phi(p^i) = p^i - p^{i-1}$
    \item $\phi(p) = p -1$ if p is prime
\end{enumerate}

Using the above properties we can solve the for any $\phi(n)$:

\begin{enumerate}[a)]
    \item
        The divisors of 63 are: 1, 3, 7, 9, 21, 63. With this information, using property 1, it is possible to determine that $\phi(63) = \phi(7)\phi(9)$. Using property 3 we can determine $\phi(7) = 7-1 = 6$. Next, using property 2, we can determine that $\phi(9)=\phi(3^2)=3^2-3=6$. Bringing this all back together we get $\phi(63) = \phi(7) \phi(9) = 6 \cdot 6 = 36$.\\

        $\therefore$ $\phi(63) = 36$
    \item
        The divisors of 246 are: 1, 2, 3, 6, 41, 82, 123, 246. Using property 1 we can break this down into $\phi(246) = \phi(41) \phi(6)$. Since 41 is prime we can use rule 3 to determine $\phi(41) = 41 - 1 = 40$. Using rule 1 followed by rule 3 we can determine $\phi(6)=\phi(2)\phi(3)=(2-1)(3-1)=2$. Bringing this all back together we get $\phi(246) = \phi(41)\phi(6) = 40 \cdot 2 = 80$.\\

        $\therefore$ $\phi(246)=80$.
    \item
        The divisors of 280 are: 1, 2, 4, 5, 7, 8, 10, 14, 20, 28, 35, 40, 56, 70, 140, 280. With this information, using property 1, it is possible to determine that $\phi(280) = \phi(35) \phi(8)$. Repeating with property 1 followed by using property 3, $\phi(35)=\phi(7)\phi(5)=(7-1)(5-1)=24$. Next, using property 2, we can determine that $\phi(8)=\phi(2^3)=2^3 - 2^2 = 4$. Bringing this all back together we get $\phi(280) = \phi(35) \phi(8) = 24 \cdot 4 = 96$.\\

        $\therefore$ $\phi(280) = 96$
    \item
        The divisors of 4220 are: 1, 2, 4, 5, 10, 20, 211, 422, 844, 1055, 2110, 4220. With this information, using property 1, it is possible to determine that $\phi(4220) = \phi(211) \phi(20)$. Using property 3 we can determine that $\phi(211) = 210$. Next, using property 1 we can determine that $\phi(20)=\phi(5)\phi(4)$. From this we can determine that, using property 3, $\phi(5)=5-1=4$ and, using property 2, $\phi(4)=\phi(2^2)=2^2-2=2$. Thus $\phi(20)=2 \cdot 4 = 8$. Bringing this all back together we get $\phi(4220) = \phi(211) \phi(20) = 210 \cdot 8 = 1680$.\\

        $\therefore$ $\phi(4220) = 1680$
\end{enumerate}

\section*{Question 2}

Node JS implementation of the Elgamal algorithm.\\

To run:

\begin{enumerate}
    \item node q2.js
    \item genkeys <- generates a private and public key
    \item encrypt <YA> <message> <- generates C1 and C2
    \item decrypt <XA> <C1> <C2> <- outputs the decrypted number
\end{enumerate}

\codefile{javascript}{q2.js}

When $X_a = 1$ and $X_A=q-1$ both of their $C_1$ values are the same. Furthermore in the case of 1, K is also equal to 1. And in the case of $q-1$ K is $q-2$. Thus a range is specified to ensure that these values aren't always used otherwise it would become trivial to derive they keys used in encryption (also why k introduces randomness).\\

Elgamal's secruity relies difficult logarithms to compute if the values of the keys are unknown. In this case alpha is a primitive root of 67, thus causing brute force calculation to be difficult. Thus since this property is true the algorithm will protect all values of M equally such that $0 \leq M \leq (q-1)$.

\section*{Question 3}

\begin{enumerate}[a)]
    \item
        277:\\

        Factor: $N = 277-1$\\
        Find $N-1=2^k \cdot m$: $276=2^2 \cdot 69$\\
        Using $a=2$ compute $b_k={(a^m)}^{2^k} \mod 277$ until $b_{k-1}$ or it is congruent to 1/-1\\

        $
            \begin{aligned}
                b_0 &: (2^{69})^{2^0} \equiv 60 \mod 277\\
                b_1 &: (2^{69})^{2^1} \equiv 276 \mod 277\\
            \end{aligned}
        $\\

        $\therefore$ 277 is likely prime since $b_1$ is congruent to -1.\\

        279:\\

        Factor: $N = 279-1$\\
        Find $N-1=2^k \cdot m$: $278=2^1 \cdot 139$\\
        Using $a=2$ compute $b_k={(a^m)}^{2^k} \mod 279$ until $b_{k-1}$ or it is congruent to 1/-1\\

        $
            \begin{aligned}
                b_0 &: (2^{139})^1 \equiv 47 \mod 279\\
            \end{aligned}
        $\\

        $\therefore$ 279 is defiantly not prime since $b_0$ is composite.\\

        281:\\

        Factor: $N = 281-1$\\
        Find $N-1=2^k \cdot m$: $280=2^3 \cdot 35$\\
        Using $a=2$ compute $b_k={(a^m)}^{2^k} \mod 281$ until $b_{k-1}$ or it is congruent to 1/-1\\

        $
            \begin{aligned}
                b_0 &: (2^{35})^{2^0} \equiv 280 \mod 281\\
            \end{aligned}
        $\\

        $\therefore$ 281 is probably prime since $b_0$ is congruent to -1.\\

        283:\\

        Factor: $N = 283-1$\\
        Find $N-1=2^k \cdot m$: $282=2^1 \cdot 141$\\
        Using $a=2$ compute $b_k={(a^m)}^{2^k} \mod 283$ until $b_{k-1}$ or it is congruent to 1/-1\\

        $
            \begin{aligned}
                b_0 &: (2^{141})^{2^0} \equiv 282 \mod 283\\
            \end{aligned}
        $\\

        $\therefore$ 281 is probably prime since $b_0$ is congruent to -1.\\

        285:\\

        Factor: $N = 285-1$\\
        Find $N-1=2^k \cdot m$: $284=2^2 \cdot 71$\\
        Using $a=2$ compute $b_k={(a^m)}^{2^k} \mod 285$ until $b_{k-1}$ or it is congruent to 1/-1\\

        $
            \begin{aligned}
                b_0 &: (2^{71})^{2^0} \equiv 143 \mod 283\\
                b_1 &: (2^{71})^{2^1} \equiv 214 \mod 283\\
            \end{aligned}
        $\\

        $\therefore$ 285 is defiantly not prime since $b_1$ is composite.\\

        287:\\

        Factor: $N = 287-1$\\
        Find $N-1=2^k \cdot m$: $286=2^1 \cdot 143$\\
        Using $a=2$ compute $b_k={(a^m)}^{2^k} \mod 287$ until $b_{k-1}$ or it is congruent to 1/-1\\

        $
            \begin{aligned}
                b_0 &: (2^{143})^{2^0} \equiv 172 \mod 283\\
            \end{aligned}
        $\\

        $\therefore$ 287 is defiantly not prime since $b_0$ is composite.\\
    \item

        Encryption using the following values:\\
        n: 78391\\
        p: 277\\
        q: 283\\
        M: 2\\
        e: 41\\

        $M^e \mod n = c$ so $2^{41} \mod 78391 = 21026$\\

        $\therefore$ the encrypted message is 21026.
    \item
        From the values above we can compute m using the totent function:\\
        m: $\phi(n)=\phi(277)\phi(283)=(277-1)(276-1)=77832$\\
        d: $41^{-1} \mod 77832 = 72137$

        Now that we have solved for d and m we are able to compute the private and public keys.

        $\therefore$ the private key is (72137, 277, 283) and the public key is (41, 78391).
    \item
        Decryption (without then with Chinese remainder theorem)
        \begin{enumerate}[i)]
            \item
                Without the Chinese remainder theorem decryption is simply just: $c^d \mod n = M$ or $21026^{72137} \mod 78391 = M = 2$. This can be done on any powerful calculator, though the Chinese remainder theorem makes it much easier to compute (as seen below).
            \item
                $d_p: d \mod (p-1) = 72137 \mod 276 = 101$\\
                $d_q: d \mod (q-1) = 72137 \mod 282 = 227$\\
                $q_{inv}: q^{-1} \mod p = 277^{-1} \mod 283 = 47$\\
                $c_p: c \mod p = 21026 \mod 277 = 251$\\
                $c_q: c \mod q = 21026 \mod 283 = 84$\\
                $m_1: (C_p)^{d_p} \mod p = 251^{101} \mod 277 = 2$\\
                $m_2: (C_q)^{d_p} \mod p  = 84^{227} \mod 283 = 2$\\
                $h: q_{inv} \cdot (m_1 - m_2) \mod p = 47 \cdot (2-2) \mod 277 = 0$\\
                $m: m_2 + h \cdot q = 2 + 0 \cdot 283$\\

               $\therefore$ Using the Chinese remainder theorem we get the message as 2.
        \end{enumerate}
\end{enumerate}


\end{document}
